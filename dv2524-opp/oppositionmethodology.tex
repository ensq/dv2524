% oppositionmethodology.tex

\paragraph{Methodology}
\label{par:methodology}
The described methodology and practices of the opposed thesis is well described, albeit - and naturally so - slightly esoteric to someone not introduced to the target subject.
In addition to describing algorithms and tools used, the dissertation methodology sections also expand upon the performed literature review; the latter which the author would like to comment on.

The author would argue that the methodology used when performing the literature review bears little academic interest to the produced results and thus to the reader of the thesis.
The removal of material not necessarily relevant would condense and thus strengthen the relevance of the thesis.
If the material is present in the thesis proposal, the author would recommend referring to said proposal - where the planned methodology for such a literary review may be more of interest.

However, due to the referenced material elaborating upon 'Threats to validity', and said literary review having affected the choice of studied algorithms and methods so greatly - one might argue that said material ought still be included in the thesis.
If so, the author would argue that the material would belong under an Appendix-chapter, since the reader is not necessarily interested in why each respective algorithm was chosen fit-for-study, until having read the report.
In this way, the material may be considered to be complementary.

% TODO:
% Is there any info as to what causes the extreme lenghts of time taken to perform the analysis?
