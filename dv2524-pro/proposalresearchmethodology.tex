% proposalresearchmethodology.tex

\chapter{Research~Methodology}
\label{cha:researchmethodology}
Proposed study is subdivided into three distinct phases.
These phases consist of a pilot~study, implementation, and experiment, respectively, with the purpose of facilitating a clear comprehension of current objectives in the presence of each phase.

The pilot~study is comprised of a case- and literary study of the task at hand, with the goal of establishing an implementational outline for the project, such as appropriate target \termos\ and viable languages for subsequent implementation of proposed components (\termeg , viable methods for reasonably fast transfer of large chunks of data in-between \termtarget\ and \termhost ), and other details surrounding the subsequent implementation.
Additionally, the pilot~study goes in-depth into previous work in the area along with a case~study of similar research.

The implementation constitutes the realization of the paravirtualized technology described in this document.
We propose an implementation in accordance to a certain \termrefsolu\ (\dvcmdrefsee{\dvcmdrefsec{sec:researchmethodology_referencesolution}}), which we hope may ease the development process.
Throughout this document, several concepts and phrases may be recognized and related to the \termandroiddesignoverview\ (\dvcmdrefappb{sec:appendixb_androidopenglesdesignoverview}), which is included in its entirety under \dvcmdrefname{cha:appendixb}.
In accordance to said \termrefimpl , this proposal suggests achieving acceleration of \termopengles ~$2.0$ by implementing an acceleration pipeline consisting of three stages (see~\dvcmdrefname{sec:appendixa_guestsystemlibraries}, \dvcmdrefname{sec:appendixa_hostrastterizationprocess}, and \dvcmdrefname{sec:appendixa_hostsystemtranslatorlibraries} under \dvcmdrefname{cha:appendixa}) that are, in turn, supported by a specialized communications channel in \termsimics\ (\dvcmdrefappa{sec:appendixa_simicspipe}).
Moreover, said communications channel, or '\termsimicspipe ', utilizes a specially devised protocol to describe an assortment of \termopengles\ and \termegl\ function calls when sent in-between the \termhost - and \termtarget\ systems (\dvcmdrefappa{sec:appendixa_simicspipeprotocol}).
The implementation of these components comprise the implementation of proposed study.
The work breakdown structure and implementational details of proposed solution are presented under \dvcmdrefname{cha:appendixa}, along with in-depth description of the pipeline stages described in this paragraph.

The experiment, the purpose of which is to analyze performance- and functional variations in-between the paravirtualized solution in \termsimics\ and the \termrefsolu , is proposed to be performed using some benchmark for the target \termapi\ (\dvcmdrefsee{\dvcmdrefsec{sec:researchmethodology_experimentmethodology}}).
In addition to measuring performance, the experiment phase concerns investigation into the benefits and flaws of paravirtualized graphics in system simulators.
As such, pursuant to the aims and objectives outlined in \dvcmdrefcha{cha:aimandobjectives}, we propose analyzing key areas of implementation along with their benefits and subsequent flaws.
These areas may include, but not necessarily be limited to, efficient communication in-between \termtarget\ and \termhost\ , execution pipeline improvements, and investigation into obstacles in the presence of virtual platform beneficiaries such as \termdetexe -, \termcheckpointing -, and \termrevexe\ functionality.

\section{Reference~Solution}
\label{sec:researchmethodology_referencesolution}
The \termandroidemu\ is described as a virtual mobile device emulator and supports virtualization of an assortment of mobile hardware permutations in the \termandroid ~\termsdk \dvpcmdciteref{web:google:2013:usingtheemulator}.
In the presence of the \termandroid ~$4.0.3$\todo{4.0.3 r2. What does r2 signify?}\ release, the \termandroid ~\termsdk\ was updated to make use of hardware-assisted \termxeightysix\ virtualization; significantly increasing performance of \termcpu -bound workloads\dvpcmdciteref{web:ducrohet:2012:afasteremulator}.
In addition to this, \termgoogle\ implemented \termopengles ~$1.1$~\&~$2.0$ hardware acceleration; offering a substantial performance boost to developers utilizing the \termopengles\ framework\dvpcmdciteref{web:ducrohet:2012:afasteremulator}.
\termgoogle 's solution, the design document of which is presented in under \dvcmdrefname{cha:appendixb} (\dvcmdrefappb{sec:appendixb_androidopenglesdesignoverview}), consists of a paravirtualized implementation which circumvents the simulated system by forwarding its \termopengles\ invocations to the \termhost\ system directly via the simulator program.

As of \termandroid ~$4.4$, the \termandroidemu\ uses \termqemu\ to simulate \termarm\ devices aiding those wishing to develop software for mobile units.
The \termandroidemu\ implementation and solution will be referred to as the \termrefimpl\ or the \termrefsolu , respectively, for the remainder of this document.
In this way, the proposed study concerns implementation of paravirtualized \termopengles , in accordance to the \termrefsolu\ implemented by \termgoogle\ with the use of \termqemu , into the \termsimics\ virtual platform.

\section{Experiment~Methodology}
\label{sec:researchmethodology_experimentmethodology}
Throughout the course of the pilot study performed for the purpose of proposed study, no suitable \termopengles ~$2.0$ benchmark - with cross-platform support for both \termandroid\ and \termxeleven ~\termlinux\ - has been found.
The best pre-existing solution\footnote{Being \termglmarktwo\ developed by \termlinaro .} encountered segmentation faults when run in a \termsimics\ \termlinux\ environment and similar problems when run in the \termandroidemu\ utilizing \termqemu.
What caused these problems is unclear but since considerable time was spent attempting to solve this issue, the decision was made to develop a benchmark specifically for the purposes of stress-testing proposed solution.

The developed benchmark ought to execute on the studied platforms; being paravirtualized \termsimics , software~rasterized \termsimics , and the \termrefsolu , respectively.
It is vital to sound experiment methodology that the tests performed throughout the benchmark focuses on critical areas and suspected bottlenecks in the implementation.
Such key points during simulation may concern a large number of relatively insignificant \termopengles\ invocations, frame-wize \termopengles\ state saving, or transferal of large chunks of data such as textures.
These areas may very well be bottlenecks of graphics paravirtualization and thus call for further investigation.
Due to the proposed study accentuating performance, and since it concerns a graphics framework often used with real-time applications, it may be viable to measure experimental data in \termframespersecond ~(\termfps ).
Alternatively, lone dispatch measurements could be used in coagency with \termframespersecond\ measurements.
Additionally, it may be favorable if, during development of said benchmark, if the software could utilize the same code-base to invoke the \termopengles -libraries - independent of platform.
