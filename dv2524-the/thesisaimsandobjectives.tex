% thesisaimsandobjectives.tex
% Chapter Aims & Objectives.

\chapter{Aims \& Objectives}
\label{cha:aimsandobjectives}
The scope of the study outlined in \dvtcmdcitefur{dissertation:nilsson:2014} consist of implementing paravirtualization of a graphics \dvttermapi\ (being \dvttermopenglestwopointo ) in the \dvttermsimics\ full system simulator developed by \dvttermintel\ and sold through \dvttermintel 's subsidiary \dvttermwindriver\ 
Said implementation adheres to a certain \dvttermreferenceimplementation , which is elaborated upon in section \dvtcmdrefname{sec:appendixa_referencesolution}.

As such, the aim concerns investigating the performance, and the feasability of extended benefits and advanced functionality, of paravirtualized graphics acceleration in a virtual platform.
Said integration entails investigating, analyzing, and developing methods and techniques for efficient communication and execution in the \dvttermsimics\ run-time environment, in addition to identifying liabilities of paravirtualized technologies in regards to \dvttermsimics\ philosophy (being high-performance and \dvttermtiming -accurate determinism and repeatability\dvtcmdcitebib{journals:aarno:2013}).
As such, this dissertation does not exclusively concern \dvttermsimics\ integration, but an investigation into paravirtualized drivers in virtual platforms.
The key aspects of such integration are as follows:

\begin{enumerate*}
	\item \label{itm:enum_aspects_performance} Performance characteristics.
	Comparing the performance of a paravirtualized solution to that of a software renderer (\dvttermsimics\ vs. \dvttermsimics ).
	Comparing the \dvttermsimics\ solution to the \dvttermreferencesolution\ to see if the implementation carried over the same benefits (\dvttermsimics\ vs. \dvttermreferencesolution ).
	\item \label{itm:enum_aspects_advancedanalyze} A naïve porting of the \dvttermreferenceimplementation 's \dvttermopengles\ acceleration into \dvttermsimics\ would not support advanced features such as \dvttermdeterministicexecution , \dvttermcheckpointing , or \dvttermreverseexecution .
	Therefore, it would be beneficial to analyze the solution in terms of what it would take to support these features.
	\item \label{itm:enum_aspects_advancedimplement} Extend the \dvttermopengles\ acceleration to support \dvttermcheckpointing\ and \dvttermreverseexecution .
\end{enumerate*}

The scope for this dissertation is to investigate (\ref{itm:enum_aspects_performance}) and (\ref{itm:enum_aspects_advancedanalyze}) with (\ref{itm:enum_aspects_performance}) being the focal point of the study and (\ref{itm:enum_aspects_advancedimplement}), or part of (\ref{itm:enum_aspects_advancedimplement}), being the stretch goal.
As such, the objectives that follow from the stated aim may be summarized as follows (see chapter \ref{cha:methodologysolution} for implementational details):

\newcommand*\objective[1]{\item}
\begin{multicols}{2}
\begin{enumerate}[(a)]
	\objective{1} Establish performance-wize sound communications between \dvttermhost\ and \dvttermtarget .
	\objective{2} Be able to send, recieve, and callback, data in-between \dvttermhost\ and \dvttermtarget .
	\objective{3} Translate given methods into desktop \dvttermapi\ calls and invoke translated methods on the \dvttermhost\ system.
	\objective{4} Display results in \dvttermtarget\ system.
	\objective{5} Analyze performance variations of paravirtualized solution in regard to existing software~rasterizer in \dvttermsimics , as described in (\ref{itm:enum_aspects_performance}).
	\objective{6} Investigate feasability of advanced functionality described in (\ref{itm:enum_aspects_advancedanalyze}).
\end{enumerate}
\end{multicols}