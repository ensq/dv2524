% presentation.tex

\documentclass{beamer}
\usetheme{Intel}
%\setbeameroption{show notes} % If annotations are desired.

% Package inclusion:
\usepackage{hyperref}
\usepackage{tikz} % Used by presentationprogressbar.tex

% Package configuration:
\hypersetup{
	colorlinks=true,
	linkcolor=blue,
	urlcolor=blue,
	citecolor=blue,
	anchorcolor=blue
}
\usetikzlibrary{calc}

\input{presentationprogressbar.tex}

\title{Paravirtualizing OpenGL ES in Simics}
\subtitle{Master's Thesis in Computer Science}
\author{Eric Nilsson}
\institute{Blekinge Institute of Technology}
\date{\today} % Fix

% Opposition presentation:
% 20 minutes
% Three key points:
% * Paravirtualization is feasible for accelerating graphics in virtual platforms.
% * Magic instructions is good methodology to transmit real-time invocations between target and host systems.
% * Paravirtualization may be subject to advanced functionality.
% The frames should lead to these conclusions.

% Screen-play:
% ---
% Front page
% Introduction:
% * (Virtualization)
% * Simics
% * [SHOW VIDEO]
% * Present key points
% Background:
% * Graphics virtualization
% Paravirtualization:
% * Solution architecture
% * (Man-in-the-middle windows)
% * Magic instructions and memory table traversal
% Experiment:
% * Benchmarks
% * Profiling methodology
% Results:
% * Chess results
% * Julia results
% * (Phong results)
% Conclusion:
% * Magic instruction overhead
% * Performance gains
% * Key points recap
% Acknowledgement
% Back page
% ---

\begin{document}
	% FRONT MATTER
	% ---
	% presentationfront.tex

\title{Paravirtualizing OpenGL ES in Simics}
\subtitle{Master's Thesis in Computer Science}
\author{Eric Nilsson}
\institute{Intel Corporation}
\date{\today} % Fix

\begin{frame}
	\titlepage
\end{frame}
	
	% BODY MATTER
	% ---
	% INTRODUCTION
	\section{Introduction}
	% Simics
	\subsection{Simics}
	% presentationsimics.tex

\begin{frame}

\frametitle{Wind River\texttrademark\ Simics\texttrademark }

\begin{itemize}
	\item Full-system simulator\note{Meaning an architectural simulator which may run an unmodified software stack.}
	\item Originally devised at SICS\footnote{The Swedish Institute of Computer Science.}\note{This was the first instance of an unmodifed OS running in an entirely simulated environment.}
	\item Developed by Intel\textregistered 
	\item Sold through Intels subsidiary Wind River Systems, Inc.
	\item Used in the industry by groups such as:
	\begin{itemize}
		\item IBM
		\item NASA
		\item Lockheed Martin
	\end{itemize}
	\item Utilized extensively in academia\footnote{$300+$ universities.}
\end{itemize}

\end{frame}

	% Demonstration
	\subsection{Demonstration}
	% presentationsimics.tex

\begin{frame}	

\frametitle{Demonstration}

Julia Benchmark
\begin{itemize}
	\item \href{http://youtube.com/embed/GKs6OlWKFV8?rel=0&vq=hd1080&autoplay=1}{[Hardware accelerated on the simulation host]}
	\item \href{http://youtube.com/embed/3sCyzppFL0w?rel=0&vq=hd1080&autoplay=1}{[Software rasterized on the simulation target]}
	\item \href{http://youtube.com/embed/__d_EeZBzwc?rel=0&vq=hd1080&autoplay=1}{[Paravirtualized on the simulation target]}
\end{itemize}

% TODO:
% Complement with performance approximation table

\end{frame}
	% Key points
	\subsection{Key points}
	% presentationkeypoints.tex

\begin{frame}
\frametitle{Key points}

\begin{block}{\#1}
	Paravirtualization is feasible for accelerating graphics in virtual platforms.
\end{block}

\begin{block}{\#2}
	Magic instructions is promising methodology to carry real-time invocations between target and host systems.
\end{block}

\begin{block}{\#3}
	Paravirtualization may be subject to deterministic execution, checkpointing, and reverse execution.
\end{block}

\end{frame}


	% BACKGROUND
	\section{Background}
	% Graphics virtualization
	\subsection{Graphics virtualization}
	% presentationgraphicsvirtualization.tex

\begin{frame}
\frametitle{Graphics virtualization}

\begin{columns}
	\column{0.5\textwidth}
	\begin{block}{GPU modeling}
		Develop a GPU model virtualizing the GPU Instruction Set Architecture (ISA).
	\end{block}
	\begin{block}{PCI passthrough}
		Utilize passthrough methodology; granting virtual systems first-hand access to host machine devices.
	\end{block}
    \column{0.5\textwidth}
    \begin{block}{Soft modeling}
    	Use advanced software rasterizers optimizing GPU kernel simulation for CPU architectures.
    \end{block}
    \begin{block}{Paravirtualization}
    	Selectively modify the virtual architecture to accomodate scalability, performance, and simplicity.
    \end{block}
\end{columns}
	
\end{frame}

	% BACK MATTER
	% ---
	% presentationback.tex

\begin{frame}

% Frame intentionally left blank.

\end{frame}

	% TODO:
	% Add colophon acknowledging progress bar.
\end{document}
