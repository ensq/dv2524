% thesismethodology.tex
% Chapter Methodology.

% Methodology:
\chapter{Methodology}
\label{cha:methodology}
\ldots

% Experiment Methodology:
\section{Experiment Methodology}
\label{sec:methodology_experimentmethodology}
Throughout the course of the pilot study performed for the purpose of this study, no suitable... See \dvtcmdcitefur{dissertation:nilsson:2014} for a demo on Further Reading sections.

% Suggested structure:
% * Solution:
% * * Component A
% * * Component B
% * Experiment Methodology:
% * * Benchmarking
% * * Measurement and Reliability

% TODO:
% | Benchmarks
% | Experiment methodology:
% | * Measurement timing and reliability on the different systems.
% | Explenation of solution and it's components.

% Segment on benchmarking in Proposal:
%Throughout the course of the pilot study performed for the purpose of proposed study, no suitable \termopengles ~$2.0$ benchmark - with cross-platform support for both \termandroid\ and \termxeleven ~\termlinux\ - has been found.
%The best pre-existing solution\footnote{Being \termglmarktwo\ developed by \termlinaro .} encountered segmentation faults when run in a \termsimics\ \termlinux\ environment and similar problems when run in the \termandroidemu\ utilizing \termqemu.
%What caused these problems is unclear but since considerable time was spent attempting to solve this issue, the decision was made to develop a benchmark specifically for the purposes of stress-testing proposed solution.

%The developed benchmark ought to execute on the studied platforms; being paravirtualized \termsimics , software~rasterized \termsimics , and the \termrefsolu , respectively.
%It is vital to sound experiment methodology that the tests performed throughout the benchmark focuses on critical areas and suspected bottlenecks in the implementation.
%Such key points during simulation may concern a large number of relatively insignificant \termopengles\ invocations, frame-wize \termopengles\ state saving, or transferral of large chunks of data such as textures.
%These areas may very well be bottlenecks of graphics paravirtualization and thus call for further investigation.
%Due to the proposed study accentuating performance, and since it concerns a graphics framework often used with real-time applications, it may be viable to measure experimental data in \termframespersecond ~(\termfps ).
%Alternatively, lone dispatch measurements could be used in coagency with \termframespersecond\ measurements.
%Additionally, it may be favorable if, during development of said benchmark, if the software could utilize the same code-base to invoke the \termopengles -libraries - independant of platform.
