% rejoinder.tex

\documentclass[a4paper,oneside]{bth}
\DeclareGraphicsExtensions{.pdf}

% Packages none of which the author of the original template has stated the purpose of:
\usepackage{url}
\usepackage{xtab}
\usepackage{pifont}
\usepackage{xspace}
\usepackage{amsthm}
\usepackage{amsmath}
\usepackage{mathenv}
\usepackage{amssymb}
\usepackage{graphicx}
\usepackage{listings}
\usepackage{multirow}
\usepackage{longtable}
\usepackage{changepage}

% User includes:
\usepackage{pdfpages} % Used to include .pdf documents in appendix.
\usepackage{datetime} % Used to format dates.
\usepackage[hidelinks]{hyperref} % Force hide hyperlink borders. Consider removing this, as such settings ought to be up to .pdf-reader.
\usepackage{alltt} % Used for verbatim environments under oppositionconclusion.

% Local packages includes:
\usepackage[footinfo]{gitinfo}

\usepackage[T1]{fontenc}
\usepackage[utf8]{inputenc}

\begin{document}
\pagestyle{plain}
\pagenumbering{roman}

% Front matter:
% rejoinderfront.tex
% Describes front pages of document.

\newcommand{\logossize}{3cm}

% ---
% Front page
% ---

% Thesis info and institutional logo:
{\pagestyle{empty}
\changepage{5cm}{1cm}{-0.5cm}{-0.5cm}{}{-2cm}{}{}{}
\noindent%
{\small
\begin{tabular}{p{0.75\textwidth} p{0.25\textwidth}}
\textit{Thesis Rejoinder}&\multirow{4}{*}{\includegraphics[trim = 3mm 0mm 3mm 4.7cm, clip, width=\logossize]{bthnotext}}\\ % Hack, but at least it's less of a hack than the previous solution of a misplaced, hard-coded, logo.
\textit{Master's Thesis in Computer Science}\\
\textit{MM YYYY}\\ % TODO: Complement.
\end{tabular}}

% Title:
\begin{center}
\par\vspace {7cm}
{\Huge\textbf{Thesis Rejoinder}} % Centered Title Times Font\\*[0.25cm] Size 24 Bold
\par\vspace {0.5cm}
{\Large\textbf{Paravirtualizing OpenGL ES in Simics}} % Centered Subtitle Times Font Size 16 Bold
\par\vspace {3cm}
{\Large\textbf{Eric Nilsson}}
\par\vspace{0.5cm}
{\large\textit{\textbf{}}}
\par\vspace{0.5cm}
{\Large\textbf{}}
\par\vspace {6cm}
\end{center}

% Institutional info and Intel logo (if enabled in build settings):
\noindent
{\small
	\begin{tabular}{p{0.75\textwidth} p{0.25\textwidth}}
	Dept. Computer Science \& Engineering&\multirow{4}{*}{}\\
	Blekinge Institute of Technology\\
	SE--371 79 Karlskrona, Sweden
	\end{tabular}}
\clearpage
}

% ---
% Additional details page
% ---

% Thesis details:
{\pagestyle{empty}
\changepage{5cm}{1cm}{-0.5cm}{-0.5cm}{}{-2cm}{}{}{}
\noindent%
\begin{tabular}{p{\textwidth}}
{\small This report is submitted to the Department of Computer Science \& Engineering at Blekinge
Institute of Technology in partial fulfillment of the requirements for the degree of Master
of Science in Computer Science.}
\end{tabular}

\par\vspace{10cm}

% Author and supervisor contact information:
\noindent%
\begin{tabular}{p{0.5\textwidth}lcl}
\textbf{Contact Information:}\\
Author(s):\\
Eric Nilsson\\
E-mail: \href{mailto:erne09@student.bth.se}{erne09@student.bth.se} \\ % Consider adding additional e-mail adress.
\par\vspace {5cm}
University advisor:\\
Prof.\ Håkan Grahn, Research Dean of Faculty\\
Dept. Computer Science \& Engineering

\par\vspace {1cm}

% Institutional contact information:
\noindent%
 \\
Dept. Computer Science \& Engineering & Internet & : & \href{http://www.bth.se/didd}{www.bth.se/didd}\\
Blekinge Institute of Technology & Phone	& : & +46 455 38 50 00 \\
SE--371 79 Karlskrona, Sweden & Fax & : & +46 455 38 50 57 \\
\end{tabular}
\clearpage
} % Back to \pagestyle{plain}

\setcounter{page}{1}


\cleardoublepage
\pagestyle{headings}
\pagenumbering{arabic}

% Body matter:
\chapter*{Introduction}
This rejoinder summarizes measures taken to improve the thesis 'Paravirtualizing OpenGL ES in Simic', in accordance to advice given by acting student opponent Alexander Mohlin.
The measures are elaborated upon in the order encountered in the opposition report (attached).

\section*{Acknowledgement}
The author would like to acknowledge the contribution of acting student opponent Alexander Mohlin, and thank the opponent for his criticism and suggestions which has helped in greatly improving my thesis.

\chapter*{Rejoinder}
Below, measures taken are presented in paragraphs - ordered in accordance to the criticism given in the opposition report.

\paragraph{Seperate discussion section}
The opponent recommends the addition of a seperate discussion section in order to increase the legibility of the material.

In response to this criticism, a Discussion chapter has been added to the thesis.

\paragraph{Background chapter}
The opponent explains that having the Background-chapter placed before the Related Work chapter may help readers not familiar with the subject to be introduced to many of the concepts prior to being elaborated upon in Related Work.

In response to this criticism, the Background chapter has been moved to prior that of Related Work.

\paragraph{Figures}
The opponent points out that a descriptory figure visualizing the paravirtualized methodology described in the thesis may aid the reader in comprehending the components and their relationships.

In response to this criticism, the author has added several figures - including an overview of the paravirtualized technology described in the document - to help facilitate understanding of the devised solution.

\paragraph{Academic terminology}
The opponent criticizes the use of the phrase 'experiment' to describe the experimental methodology used to reach the conclusions of the thesis; claiming that dependant and independant variables, objects and hypotheses (commonly defined for 'controlled experiments') are required for this phrase to be used.

In response to this, and in order to clarify, the author has added a complimentary paragraph to chapter Experimental Methodology to relate to the reader that the phrase 'experiment' is in no way meant to signify utilization what is commonly known as a 'controlled experiment', or indeed any other pre-determined experiment type.
The phrase is merely used to describe the experimental methodology used throughout the course of the dissertation, which is similar to the methodologies used in similar studies concerning system simulation.
To ensure this does not cause confusion, the clarification appears before the phrasing has occurred in the document.

\paragraph{Significance test}
The opponent suggests the addition of a significance test to strengthen the conclusions drawn in the thesis.

The addition of a significance test would surely strengthen the conclusions of the thesis.
However, the author would like to argue that the produced results, and the conclusions drawn from said results, are not misrepresented and do not require the addition of a significance test.
Although the author would gladly add such a test, other improvements have been prioritized since significance tests are not commonplace for studies such as this one.
Convinced that the addition of such a test would further increase the validity of the thesis, the author has added this to a list of future work, and hopes to add this to later revisions of the thesis.

\paragraph{Advanced functionality}
The opponent points out that, although being listed as research questions, some question formulations are not discussed until the thesis appendix.

In order to correct this, the author has broken up the research questions into an additional category; being discussion questions.
Furthermore, the discussion of these questions has been moved from the appendix into a Discussion chapter, as advised by the thesis opponent.
In this way, these question formulations are treated before the thesis conclusion

\paragraph{Grammar}
The opponent identifies several mistakes in grammar.

The author has attempted to correct these to the best of his ability.

\chapter{Appendix}
\includepdf[pages={-}]{opp_alex_eric.pdf}

% Back matter:
\include{rejoinderback}

\end{document}
